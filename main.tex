\documentclass[9pt, letterpaper]{extarticle}

% Packages:
\usepackage[
    ignoreheadfoot, % set margins without considering header and footer
    top=1.4 cm, % seperation between body and page edge from the top
    bottom=1.4 cm, % seperation between body and page edge from the bottom
    left=1.4 cm, % seperation between body and page edge from the left
    right=1.4 cm, % seperation between body and page edge from the right
    footskip=1.0 cm, % seperation between body and footer
    % showframe % for debugging
]{geometry} % for adjusting page geometry
\usepackage{titlesec} % for customizing section titles
\usepackage{tabularx} % for making tables with fixed width columns
\usepackage{array} % tabularx requires this
\usepackage[dvipsnames]{xcolor} % for coloring text
\definecolor{primaryColor}{RGB}{0, 0, 0} % define primary color
\usepackage{enumitem} % for customizing lists
\usepackage{fontawesome5} % for using icons
\usepackage{amsmath} % for math
\usepackage[
    pdftitle={Pradyumn Tendulkar's Resume},
    pdfauthor={Pradyumn Tendulkar},
    pdfcreator={LaTeX with RenderCV},
    colorlinks=true,
    linkcolor=blue,
    pdfborder={0 0 1},
    urlcolor=blue
]{hyperref} % for links, metadata and bookmarks
\usepackage[pscoord]{eso-pic} % for floating text on the page
\usepackage{calc} % for calculating lengths
\usepackage{bookmark} % for bookmarks
\usepackage{lastpage} % for getting the total number of pages
\usepackage{changepage} % for one column entries (adjustwidth environment)
\usepackage{paracol} % for two and three column entries
\usepackage{ifthen} % for conditional statements
\usepackage{needspace} % for avoiding page brake right after the section title
\usepackage{iftex} % check if engine is pdflatex, xetex or luatex

% Ensure that generate pdf is machine readable/ATS parsable:
\ifPDFTeX
    \input{glyphtounicode}
    \pdfgentounicode=1
    \usepackage[T1]{fontenc}
    \usepackage[utf8]{inputenc}
    \usepackage{lmodern}
\fi

\usepackage{charter}

% Some settings:
\raggedright
\AtBeginEnvironment{adjustwidth}{\partopsep0pt} % remove space before adjustwidth environment
\pagestyle{empty} % no header or footer
\setcounter{secnumdepth}{0} % no section numbering
\setlength{\parindent}{0pt} % no indentation
\setlength{\topskip}{-5pt} % no top skip
\setlength{\columnsep}{0.15cm} % set column seperation
\pagenumbering{gobble} % no page numbering

\titleformat{\section}{\needspace{4\baselineskip}\bfseries\large}{}{0pt}{}[\vspace{1.5pt}\titlerule]

\titlespacing{\section}{
    % left space:
    0pt
}{
    % top space:
    0.11 cm
}{
    % bottom space:
    0.11 cm
} % section title spacing

\renewcommand\labelitemi{$\vcenter{\hbox{\small$\bullet$}}$} % custom bullet points
\newenvironment{highlights}{
    \begin{itemize}[
        topsep=0.10 cm,
        parsep=0.10 cm,
        partopsep=0pt,
        itemsep=0pt,
        leftmargin=0 cm + 10pt
    ]
}{
    \end{itemize}
} % new environment for highlights


\newenvironment{highlightsforbulletentries}{
    \begin{itemize}[
        topsep=0.15 cm,
        parsep=0.15 cm,
        partopsep=0pt,
        itemsep=0pt,
        leftmargin=5pt
    ]
}{
    \end{itemize}
} % new environment for highlights for bullet entries

\newenvironment{onecolentry}{
    \begin{adjustwidth}{
        0 cm + 0.00001 cm
    }{
        0 cm + 0.00001 cm
    }
}{
    \end{adjustwidth}
} % new environment for one column entries

\newenvironment{twocolentry}[2][]{
    \onecolentry
    \def\secondColumn{#2}
    \setcolumnwidth{\fill, 4.5 cm}
    \begin{paracol}{2}
}{
    \switchcolumn \raggedleft \secondColumn
    \end{paracol}
    \endonecolentry
} % new environment for two column entries

\newenvironment{threecolentry}[3][]{
    \onecolentry
    \def\thirdColumn{#3}
    \setcolumnwidth{, \fill, 4.5 cm}
    \begin{paracol}{3}
    {\raggedright #2} \switchcolumn
}{
    \switchcolumn \raggedleft \thirdColumn
    \end{paracol}
    \endonecolentry
} % new environment for three column entries

\newenvironment{header}{
    \setlength{\topsep}{0pt}\par\kern\topsep\centering\linespread{1.5}
}{
    \par\kern\topsep
} % new environment for the header

\newcommand{\placelastupdatedtext}{% \placetextbox{<horizontal pos>}{<vertical pos>}{<stuff>}
  \AddToShipoutPictureFG*{% Add <stuff> to current page foreground
    \put(
        \LenToUnit{\paperwidth-3 cm-0 cm+0.05cm},
        \LenToUnit{\paperheight-1.0 cm}
    ){\vtop{{\null}\makebox[0pt][c]{
        \small\color{gray}\textit{Last updated in February 2026}\hspace{\widthof{Last updated in February 2026}}
    }}}%
  }%
}%

% save the original href command in a new command:
\let\hrefWithoutArrow\href

% new command for external links:


\begin{document}
    \newcommand{\AND}{\unskip
        \cleaders\copy\ANDbox\hskip\wd\ANDbox
        \ignorespaces
    }
    \newsavebox\ANDbox
    \sbox\ANDbox{$|$}

    \begin{header}
    \vspace{-2cm}
        \fontsize{20 pt}{20 pt}\selectfont Pradyumn Tendulkar


        \normalsize
        \mbox{Boston, MA}%
        \kern 5.0 pt%
        \AND%
        \kern 5.0 pt%
        \mbox{\hrefWithoutArrow{mailto:pktendulkar@wpi.edu}{pktendulkar@wpi.edu}}%
        \kern 5.0 pt%
        \AND%
        \kern 5.0 pt%
        \mbox{\hrefWithoutArrow{tel:+1 7745190324}{774-519-0324}}%
        \kern 5.0 pt%
        \AND%
        \kern 5.0 pt%
        \mbox{\hrefWithoutArrow{https://www.linkedin.com/in/p-tendulkar}{https://www.linkedin.com/in/p-tendulkar}}%
        \kern 5.0 pt%
         \AND%
         \kern 5.0 pt%
         \mbox{\hrefWithoutArrow{https://github.com/pradyten}{https://github.com/pradyten}}%
    \end{header}

    \vspace{5 pt - 0.3 cm}

    \section{Education}

        \begin{twocolentry}{
            Graduation Date: Dec 2025
        }
            \textbf{Worcester Polytechnic Institute, Worcester, MA}
        \end{twocolentry}
        \vspace{0.10 cm}
        \begin{onecolentry}
            MS in Data Science (GPA: 4.0)\\
            Courses: Deep Learning, Natural Language Processing, Statistical Methods for Data Science, MLOps, Business Intelligence
        \end{onecolentry}

        \vspace{0.10 cm}

        \begin{twocolentry}{
            Aug 2021 - May 2024
        }
            \textbf{Vellore Institute of Technology, India}
        \end{twocolentry}
        \vspace{0.10 cm}
        \begin{onecolentry}
            BTech in Computer Science and Engineering (GPA: 3.56)\\
            Courses: Python, Machine Learning, Natural Language Processing, Software Engineering, Database Management Systems
        \end{onecolentry}

        \vspace{0.10 cm}


    \section{Skills}

        \begin{onecolentry}
            \textbf{Languages:} Python, C/C++, SQL, TypeScript, R \\
            \textbf{ML \& NLP:} PyTorch, TensorFlow, Scikit-learn, XGBoost, Hugging Face, Sentence-Transformers, spaCy, NLTK, Gensim, Keras \\
            \textbf{LLM \& GenAI:} OpenAI API, Claude API, LangChain, RAG, Fine-Tuning, Prompt Engineering, Agentic AI, Vector Databases \\
            \textbf{Cloud \& MLOps:} AWS (Bedrock, Lambda, Step Functions, SageMaker, Textract, Comprehend), GCP, Docker, Kubernetes, W\&B, CI/CD \\
            \textbf{Data \& Tools:} Pandas, NumPy, Pydantic, FastAPI, Streamlit, PostgreSQL, MongoDB, Git
        \end{onecolentry}

        \vspace{0.10 cm}


    \section{Experience}

        \begin{twocolentry}{
            Aug 2025 – Dec 2025
        }
            \textbf{AI Engineer}, CrossingLegal, Boston, MA
        \end{twocolentry}

        \vspace{0.10 cm}
        \begin{onecolentry}
            \begin{highlights}
                \item Architected multi-agent validation pipeline using \textbf{AWS Step Functions}, \textbf{Lambda}, and \textbf{Bedrock Claude API} for vision-based PDF extraction into structured JSON, reducing H-1B processing from 5 days to 10 minutes.
                \item Built RAG fallback system with \textbf{Titan embeddings}, in-memory vector store, and \textbf{cosine similarity retrieval} to resolve ambiguities across 50+ Form I-129 fields, achieving 95-99\% extraction accuracy.
                \item Developed LLM validation layer using \textbf{OpenAI}, \textbf{Pydantic} schema enforcement, and \textbf{Textract}-based KB construction, cutting manual verification effort by 70-90\%.
            \end{highlights}
        \end{onecolentry}

        \vspace{0.2 cm}

        \begin{twocolentry}{
            Jun 2025 – Aug 2025
        }
            \textbf{AI Engineer Intern}, Boredm LLC, Phoenix, AZ
        \end{twocolentry}

        \vspace{0.10 cm}
        \begin{onecolentry}
            \begin{highlights}
                \item Fine-tuned \textbf{GPT-4o-mini} via \textbf{OpenAI API} on 1000+ balanced examples for 58-entity NER extraction, implementing \textbf{Pydantic} schema validation to achieve 99.1\% production accuracy on soil lithology descriptions.
                \item Engineered synthetic data pipeline using \textbf{Python-TypeScript} hybrid architecture with dependency-aware generation, keyboard-proximity corruption, and 12 structural variants, reducing manual annotation time by 90\%.
            \end{highlights}
        \end{onecolentry}

        \vspace{0.2 cm}

        \begin{twocolentry}{
            Oct 2024 – May 2025
        }
            \textbf{Research Assistant}, Worcester Polytechnic Institute, Worcester, MA \hrefWithoutArrow{https://aisel.aisnet.org/amcis2025/data_science/sig_dsa/8/}{(Publication: Link)}
        \end{twocolentry}

        \vspace{0.10 cm}
        \begin{onecolentry}
            \begin{highlights}
                \item Developed zero-shot summarization framework using \textbf{GPT-4}, \textbf{Gensim GloVe embeddings}, and \textbf{NLTK} preprocessing to generate abstractive summaries from 5000+ Amazon reviews, published at \textbf{AMCIS 2025}.
                \item Achieved 0.9417 cosine similarity with Amazon's proprietary summaries using \textbf{scikit-learn} cosine similarity, \textbf{iterative prompt refinement}, and \textbf{BERTScore} validation on \textbf{Playwright}-scraped review data.
            \end{highlights}
        \end{onecolentry}

        \vspace{0.2 cm}

        \begin{twocolentry}{
            May 2023 – Jul 2023
        }
            \textbf{AI Engineer Intern}, SmartStream Technologies -- India
        \end{twocolentry}

        \vspace{0.10 cm}
        \begin{onecolentry}
            \begin{highlights}
                \item Built end-to-end NER pipeline using \textbf{AWS Textract}, \textbf{SageMaker Ground Truth} (custom annotation template), and \textbf{Comprehend Custom Entity Recognition} to extract entities from native PDF/Word documents, achieving 99\% accuracy.
                \item Replaced legacy rule-based extraction with \textbf{Comprehend ML workflows} and \textbf{CloudFormation}-deployed annotation infrastructure, reducing entity extraction cycles from 4 months to 3 weeks (80\% improvement).
            \end{highlights}
        \end{onecolentry}

        \vspace{0.10 cm}


    \section{Projects}

        \begin{twocolentry}{}
            \textbf{Generalizable Vision-to-JSON Extraction Framework} (\hrefWithoutArrow{https://huggingface.co/spaces/pradyten/pdf-extractor}{Live Demo})
        \end{twocolentry}

        \vspace{0.10 cm}
        \begin{onecolentry}
            \begin{highlights}
                \item Built vision-based PDF extraction system using \textbf{OpenAI Vision API} (GPT-4.1), \textbf{pypdfium2}, and \textbf{Pillow} with adaptive DPI rendering (300/145/110 based on page count), deployed on \textbf{Hugging Face Spaces} with \textbf{Streamlit} UI.
                \item Designed template-driven extraction engine supporting 12 document types (I-129, passports, resumes, transcripts) with JSON schema enforcement, achieving 100\% structural compliance across immigration and identity documents.
            \end{highlights}
        \end{onecolentry}

        \vspace{0.10 cm}

        \begin{twocolentry}{}
            \textbf{Predictive Modeling of Water Turbidity Using Remote Sensing Indices}
        \end{twocolentry}

        \vspace{0.10 cm}
        \begin{onecolentry}
            \begin{highlights}
                \item Co-authored paper (accepted at \textbf{ICFAiSE 2025}) disproving linear correlation assumptions between water depth (NDWI) and turbidity (NDTI) through 2-year time-series analysis of Indian lakes using \textbf{ARDL models}, \textbf{Random Forest}, and \textbf{XGBoost}.
                \item Achieved R\textsuperscript{2} = 0.98 using \textbf{LSTM} (\textbf{TensorFlow/Keras}, \textbf{PyTorch}) for turbidity prediction, outperforming ARIMA and linear regression baselines on sliding-window time-series frameworks.
            \end{highlights}
        \end{onecolentry}



\end{document}
